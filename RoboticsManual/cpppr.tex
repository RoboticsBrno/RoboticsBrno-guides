

\label[cpppr] \sec C++ -- příklady 

Ve všech příkladech níže je uveden vždy obsah souboru {\it main.cpp}. Text předpokládá, že nad příklady budete samostatně přemýšlet a~učit se z~nich, proto se to, co bylo řečeno u~prvního příkladu, už neopakuje u~druhého.  
Doporučuji projít soubory {\it LearningKit.h}
a~{\it LearningKit.cpp} (viz
v~\link[ref:explorer]{\Magenta}{Exploreru}\Black
adresář {\it.piolibdeps/ArduinoLearningKitStarter\_ID1745/src} ), protože jsou v~nich zkratky typu {\tt L\_R} a~jejich přiřazení pinům čipu.
Další příklady jsou na \url{http://wall.robotikabrno.cz} a~\url{https://www.arduino.cc/reference/en/}.  

\label[cpppr1] \secc Blikání LED

Program bliká červnenou \link[ref:LED]{\Magenta}{LED}\Black .   

\hisyntax{C} \begtt 
\#include "LearningKit.h"
void setup() // this part run once
\{
    pinMode(L_R, OUTPUT); // initialize LED digital pin as an output.
\}
void loop() // this part works in cycle 
\{
    digitalWrite(L_R, HIGH); // switch on red LED
    delay(500); // pause 500 miliseconds
    digitalWrite(L_R, LOW);  // switch off red LED
    delay(500);
\}
\endtt

\label[cpppr2] \secc LED zapínaná tlačítkem 

Žlutá LED zapínaná tlačítkem.  

\hisyntax{C} \begtt 
\#include "LearningKit.h"
void setup() 
\{
    pinMode(L_Y, OUTPUT); // initialize LED digital pin as an output
    
    digitalWrite(L_Y, HIGH);
\}
void loop() 
\{
    if ((digitalRead(SW1)) == LOW)
        {digitalWrite(L_Y, LOW);} 
    else {digitalWrite(L_Y, HIGH);}
  
\}
\endtt

\label[cpppr3] \secc Nejjednodušší PWM  

\link[ref:PWM]{\Magenta}{PWM}\Black \ii PWM umožňuje (ve spolupráci
s~\link[ref:PWM]{\Magenta}{drivery}\Black ) řídit motory, serva a~podobně. Zde je použito na stmívání LED pomocí potenciometru.  

\hisyntax{C} \begtt 
\#include "LearningKit.h"
void setup() 
\{
    ledcSetup(0, 1000, 10); 
    ledcAttachPin(L_Y, 0);
\}
void loop() 
\{
    ledcWrite(0, analogRead(POT1)/4);
\}
\endtt

je potřeba setupLeds(); ?? 


// ledcSetup(channel, freq, resolution)
    // channel = 0 - 15
%    // resolution = 10 => 2^10 => 1024
tento kód funguje pro čip ESP32. Pro čipy řady ATMega, které jsou na deskách Arduino uno a~Arduino nano, je potřeba použít tento kód: 
 
\rfc{dodělat program PWM}
Funguje stejně, ale místo příkazu ledcWrite je použit:  
%%%%%%%%%%%%%%%%%%%%%%%%%%%%%%%%%%%%%%%%%%%%%%%%%
