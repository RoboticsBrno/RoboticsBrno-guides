
\subsection{Rozšíření do VSCode}

\subsubsection*{Barevné zvýraznění syntaxe}

Rozšíření \textit{Bracket Pair Colorized 2} zajistí barevné odlišení dvojic závorek.

\subsubsection*{Automatické formátování}

Doplnění řádků 
\begin{verbatim}
{
	"editor.formatOnSave": true
}
\end{verbatim}

do souboru settings.json (soubor může být potřeba napřed vytvořit) do adresáře.vscode v rámci projektu zajistí, že se při každém uložení bude automaticky správně formátovat zdrojový kód. 
Přitom v hlavním adresáři projektu musí být umístěn soubor \textit{.clang-format}.
Podrobnosti \href{https://github.com/RoboticsBrno/rb_clang_format}{zde}.

%todo
workspace - rozložení pracovní plochy

- když chci ukončit terminál ve VSCode, klepnu myší do okna terminálu a pak Ctrl+C

F12 skočí na definici proměnné, na které je kurzor  

Ctrl+Alt+B - build 
Ctrl+Alt+U - upload
Ctrl+Alt+T - spusť task 


Barevné zvýraznění syntaxe ve Visual studio code: 
Bracket Pair Colorizer 2
%*************************************************

- vývojářský deník 
- zavést používání vývojových diagramů (u větších projektů (nad 100 řádek) striktně) 

- zavést zásady čitelného psaní kódu 
-- na každém řádku pouze jeden příkaz 
-- používat pouze znaky anglické abecedy, číslice a podtržítko, a to i v komentářích 
-- název proměnné nesmí začínat číslicí 
-- psát samovysvětlující kód (vhodné názvy proměnných, přehledná struktura) 
-- tabulátor 4 znaky 

- jak se ve VSCode provede změna názvu proměnné na všech místech a souborech najednou? 

snippets - nápověda ke kódu 
Ve visual studio Code:  
File > Preferences > user snippets 




naučit sebe i studenty verzovat -> Github, a to striktně od samého začátku všechny projekty (jak to udělat u legových? )

- šablonu pro postery z medvědice -> doučit zoner calisto 
-- sepsat pravidla pro vyrábění posterů + umět rozřezat plakát tak, aby se dal vytisknout na listy A3 (lze vytisknout ve škole?)
--- bílé pozadí a černý hlavní text dostatečně velký, ideálně v bodech 
--- ideálně Čj+Aj nebo dva postery  
--- přiměřeně velké fotografie (max. A4)


--------------

Octopart is the free search engine for electronic parts.
With datasheets, CAD models, and over 200 distributors,
Octopart is the best online resource for part data.

octopart.com


