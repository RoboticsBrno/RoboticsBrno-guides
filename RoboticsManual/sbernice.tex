\section{Sběrnice}


\textbf{Sběrnice} \index{sběrnice} neboli komunikační rozhraní jsou domluvené postupy/systémy, jak se dva čipy nebo dvě různá zařízení dorozumívají mezi sebou. 
Sběrnice jsou různých typů, pro naše účely stačí znát \index{UART} 
\href{https://maly.gitbooks.io/hradla-volty-jednocipy/23_seriova_komunikace/231_seriova_sbernice_spi.html}{SPI}, 
\href{https://maly.gitbooks.io/hradla-volty-jednocipy/23_seriova_komunikace/232_seriova_sbernice_i2c.html}{I2C}  
a \href{https://maly.gitbooks.io/hradla-volty-jednocipy/23_seriova_komunikace/235_rs-232,_uart,_serial.html}{USART/UART}

\subsection{USART/UART} \label{uart}

Sběrnice UART je nejjednodušší pro propojení dvou zařízení. Na začátku se musí na obou zařízeních nastavit stejná rychost přenosu 
a další parametry. Příklad programu pro UART je v \cite[strana~144]{ard}. Další příklad je v kapitole \ref{prog:lorris}.

\subsection{I2C} \label{i2c} \label{spi} \index{I2C} \index{SPI}

Rozhraní \textbf{I2C}  umí na pomocí dvou drátů připojit k čipu až 127 zařízení. Komunikovat s čipem může vždy pouze jedno, ostatní tzv. \uv{poslouchají} na lince, až na ně přijde řada. 
Sběrnice I2C se neosvědčila tam, kde je větší vzdálenost mezi zařízeními, než desítky cm, ideální je, když jsou všechna komunikující zařízení na jedné (řídící) desce. 
Pro větší vzdálenosti mezi zařízeními je vhodná sběrnice \textbf{SPI}. 
Příklad programu pro I2C je v \cite[strana~152]{ard}.

%\subsection{SPI}

