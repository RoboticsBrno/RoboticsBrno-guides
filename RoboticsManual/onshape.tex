
\section{Onshape}

\subsection{Úvod, přihlášení, nový dokument}

\textbf{Onshape} \index{Onshape}
 [onšejp] je relativně jednoduchý CAD program pro navrhování 3D modelů. 
 
 \subsubsection{Možnosti použití}
 
 K čemu je pro nás Onshape dobrý? Můžeme v něm vyrobit: 
 
 \begin{description}
 	\item[--] hrubý návrh robota bez uvedení rozměrů pro debatu o konstrukci: \uv{bude to vypadat asi takhle a dělat asi tohle}
 	\item[--] podrobný návrh včetně všech rozměrů a výkresů pro vypálení dílů na laseru  
 	\item[--] cokoliv mezi tím
 \end{description}
 
 
Onshape je pro vzdělávací účely zdarma s tím, že všechno, co si v něm vytvoříte, je veřejně dostupné. 
Je dostupný přes webový prohlížeč (Opera, Firefox, Chrome) a proto funguje na všech operačních systémech. 
Podmínka je, aby prohlížeč měl zprovozněné WebGL rozhraní, což některé staré grafické karty nezvládají. 

Onshape je pouze anglicky, překlad některých pojmů je dále v textu.

Pokud s Onshape začínáte, je nutné si vytvořit účet. 
Na webu \url{onshape.com} klikněte na {\tt Sign in} (přihlášení) a po otevření přihlašovacího okna na {\tt Sign up} (založení nového účtu).


\subsubsection{Nový dokument}


\textbf{Dokument} je v Onshape obálka pro všechny soubory, které se týkají daného projektu. 

 \textbf{Projekt} je pro nás například konstrukce nového robota -- 3D model robota složený z jednotlivých dílů, vazby mezi těmito díly a výkresy všech dílů. 
 
 Po přihlášení do Onshape klepněte vlevo nahoře na {\tt Create} a zvolte {\tt Document...} 

Zadejte název dokumentu (použijte pouze anglická písmena a číslice!) a potvrďte {\tt OK}.  

Pozn.: Stejně tak při pojmenovávání čehokoliv dalšího \textbf{používejte pouze anglická písmena a číslice}.

\subsubsection{Popis pracovního prostředí}

Otevře se hlavní okno programu. 
Nahoře jsou ikony pro úpravy dílů. 
Vlevo je panel se seznamem všech geometrických prvků v projektu (díly, skicy, pomocné roviny atd.). 
Uprostřed jsou tři hlavní roviny a počátek souřadnic (origin). 
Vpravo spíše nahoře je \uv{kostka}, která ukazuje, jak je vytvářený díl nebo sestava právě otočená. 


\subsubsection{Návody}

Vpravo nahoře je tlačítko {\tt Learning Center}. Obsahuje velké množství krátkých videí, které vás programem krok za krokem provedou. 
\href{https://learn.onshape.com/courses/fundamentals-navigating-onshape}{Videa} jsou pouze anglicky, ale dobře srozumitelná. Pod každým videem je napsané všechno, co je ve videu řečeno. Pokud jenom trochu umíte anglicky, doporučuji je shlédnout, dá vám to hodně. 

Kompletní přehledná nápověda k Onshape je  \href{https://cad.onshape.com/help/Content/gettingstarted.htm?tocpath=Desktop\%20Help\%7C_____3}{zde}.

%\subsubsection{Cizí zdroje}

%Public - veřejné dokumenty 

%Když chci použít a upravit veřejný projekt někoho jiného, dám Copy dokument

%Dále viz kapitola \ref{sestava:vyroba}

\subsubsection{Označení}

Označení provedete kliknutím myši a tažením. Vytvoří se obdélník. Když táhnete myší zleva doprava, označí se pouze to, co je zcela uvnitř obdélníka. 
Když táhnete zprava doleva, označí se vše, co je alespoň částečně uvnitř obdélníka. Odznačení všech označených dílů zajistí mezerník (klávesa {\tt Space}). 


\subsubsection{Posunutí, otočení a přiblížení}

Posunutí prvku: {\tt Shift}  + šipka směru, kam chceme posouvat nebo {\tt Ctrl} + stisknuté kolečko myši.  

Prvky je možné přiblížit nebo oddálit otáčením kolečka myši. Přitom se přibližujeme k bodu, na který právě myš ukazuje. 

Stisknutím kolečka a posunem myši se otáčí dané prvky. 

Chování popsané zde odpovídá nastavení \texttt{SolidWorks} (to doporučujeme, protože SolidWorks budete časem probírat). Nastavení lze změnit, když kliknete vpravo nahoře na svoje jméno, 
	zvolíte \texttt{My account}, vlevo \texttt{Preferences} a níže na stránce \texttt{View settings}.  


\subsubsection{Plocha, objem, hmotnost}

V části díly (Parts) vlevo dole označíte díly. Vpravo dole se objeví ikona \uv{váhy}. 
Klepnutím na ni se otevře okno, kde je spočtená plocha povrchu (Surface area) a objem (Volume) označených dílů. 
Pokud je zadaný materiál (klikněte pravým tlačítkem na díl v seznamu vlevo dole, z menu vyberte {\tt Assign material...}), zobrazí se i hmotnost (mass) a další parametry.


\subsection{Postup práce v Onshape}
%Postup tvorby modelu / Postup vytváření podkladů pro výrobu robota
 
\begin{enumerate}
	\item pro každý díl vytvoříte skicu ve 2D -- kapitola \ref{skica:vyroba}
	
	\item ze skicy vytvoříte díl ve 3D -- kapitola \ref{dil:vyroba}
	
	\item díly poskládáte do sestavy a zkontrolujete, že k sobě správně pasují -- kapitola \ref{sestava:vyroba}
	
	\item vytvoříte DWG soubor ze všech dílů $\rightarrow$ podklad pro řezání na laseru -- kapitola \ref{laser:vykresy}	
\end{enumerate}


\subsection{Výroba skici} \label{skica:vyroba}


Skica (Sketch) \index{skica} je dvourozměrný podklad pro tvorbu dílů ve 3D. 

Klepnutím zvolte rovinu, ve které chcete skicu vytvářet. Klepnutím na tlačítko {\tt Sketch} vlevo nahoře založte ve zvolené rovině novou skicu. 
Zároveň se ikony nahoře změní z ikon pro úpravy dílů na ikony pro úpravy skici. Zvolte z nich např. kružnici, klepněte na počátek a tažením vytvořte kružnici na skice. 
Pomocí dalších ikon lze vytvořit mnohem složitější tvary -- více v kapitole \ref{skica:možnosti}.

Pomocí \uv{kostky} vpravo nahoře zvolte vhodnou orientaci skici. 

Abychom mohli ze skici nebo modelu vyrobit výkres, musí být tzv. úplně určená.

Skica je \textbf{úplně určená}, když má zadané všechny rozměry a také polohu (vzdálenost) od počátku nebo od bodu nebo čáry, která je vztažená k počátku.
Úplně určená skica zčerná, do té doby má okraje modré.  

Rozměry zadáváte pomocí ikony \uv{kóta}.
Při zadávání rozměrů a polohy se velmi doporučuje využívat vazeb -- rovnoběžnost, vodorovný směr, svislý směr, osová souměrnost, lineární a kruhové pole. 

Rozměry se vloží tak, že se číslo napíše ihned po dokončení daného geom. prvku do skici 
nebo se může doplnit později po kliknutí na ikonu kóty (dimensions).

Rozměry se průběžně zobrazují vpravo dole po klepnutí na prvek, jehož rozměr nás zajímá. 

Hotovou nebo rozpracovanou skicu uzavřeme pomocí zeleného zatržítka. 

Mimochodem, Onshape nemá Save $\rightarrow$ vše je automaticky ukládáno do cloudu.

\subsection{Výroba dílu  -- poznámky} \label{dil:vyroba}

Ze skici vytvoříte díl (part) ve 3D, nejčastěji pomocí příkazu vytažení (Extrude). 

Nový díl založíte pomocí tlačítka \texttt{+} vlevo dole. Z menu vyberete {\tt Create Part Studio}. Otevře se nová záložka, ve které pomocí skici začnete tvořit nový díl.  

U skic lze při výrobě dílu průběžně zapnout a vypnout viditelnost.

Barva dílu: klikněte na díl (seznam vlevo) pravým tlačítkem a  z menu vyberte {\tt Edit appearance}. 

\subsection{Výroba sestavy} \label{sestava:vyroba}

Díly, skicy, plochy nebo sestavy se vkládají pomocí tlačítka insert vlevo nahoře. 
První díl se musí ručně upevnit (fix) vůči počátku sestavy. Klikněte pravým tlačítkem na díl a zvolte {\tt fix}. 

Můžete vkládat díly, skici i plochy, svoje i kohokoliv jiného. 
Jde vložit i díly vymodelované v SolidWorks (přípona .SLDPRT). 
Tyto se napřed musí importovat (klik na {\tt Onshape} vlevo nahoře, potom na {\tt Create } pod ním a zvolit {\tt Import files...}) 
Importovaný díl vytvoří vlastní dokument. Potom se musí u tohoto dokumentu vytvořit alespoň jedna verze (poklepání otevřete dokument, vlevo nahoře mezi {\tt Onshape} 
a názvem projektu jsou tři ikony, klikněte na prostřední a použijte tlačítko {\tt Create version} ). 

Další instance stejného dílu: vlevo v soupisu dílů klikněte pravým tlačítkem na díl a zvolte {\tt Copy}. 
Potom klikněte opět pravým tl. do plochy, kde tvoříte sestavu a zvolte {\tt Paste}. Můžete taky použít klasické {\tt Ctrl+C, Ctrl+V}. 

Pohled na díly v řezu (section view): klikněte na \uv{malou kostku vpravo} a vyberte {\tt Turn section view on}, následně vyberte rovinu, podle které má řez probíhat. 

Přehled možných vazeb mezi díly (mate) je \href{https://cad.onshape.com/help/Content/mate.htm?TocPath=Desktop%20Help|Assemblies|Mates|_____0}{zde}.


\subsection{Výroba výkresů/příprava pro laser} \label{laser:vykresy}


 přibude ... 




\subsection{Možnosti skici} \label{skica:možnosti}


Některé možnosti při úpravě skici: 

trim -- vystřihnutí dané křivky "od bodu k bodu"

fillet -- zaoblení 

offset -- zdvojení hran a jejich odsazení 

mirror -- zrcadlení = osová souměrnost - nejdřív se vybírá osa, potom čáry, které se mají zrcadlit

linear pattern -- dvourozměrné lineární pole 
pod ním je ještě kruhové pole a otočení/transformace 


\subsection{Slovníček pro Onshape}

assembly -- sestava nebo podsestava, například celý robot nebo podvozek

constrain -- vazba v rámci skici

dimensions -- kóty = rozměry 

extrude -- vytažení 

fillet -- zaoblení 

linear pattern -- dvourozměrné lineární pole 

mate -- vazba v rámci sestavy (složené z dílů)

mirror -- zrcadlení = osová souměrnost - nejdřív se vybírá osa, potom čáry, které se mají zrcadlit

offset -- zdvojení hran a jejich odsazení 

origin -- počátek soustavy souřadnic

part -- díl = součástka

part studio -- tady se vytváří nové součástky 

sketch -- skica = nákres 

trim -- vystřihnutí dané křivky "od bodu k bodu"







