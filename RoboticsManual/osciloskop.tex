

\section{Osciloskop} \label{osciloskop} \index{osciloskop}

\subsection{Než začnete}

%pokud někdo nastaví jako jazyk korejštinu a spol: tl Help, potom třetí tl. zleva pod obrazovkou vyvolá nabídku jazyků \rightarrow zvolím zpět angličtinu 

\textbf{Osciloskop} měří velmi rychle napětí. Dokáže si napětí pamatovat, zobrazit závislost napětí na čase a v zobrazeném průběhu napětí je možné změřit řadu parametrů, např. frekvenci. 


Máme možnost pracovat s digitálním paměťovým osciloskopem
\href{https://www.keysight.com/en/pdx-x201837-pn-DSOX2024A/oscilloscope-200-mhz-4-analog-channels?nid=-32542.1150190\&cc=CZ\&lc=eng\&pm=ov}{Agilent/Keysight DSO-X 2024A}.
Tento osciloskop má čtyři analogové vstupy (=kanály), takže lze současně měřit a zobrazovat čtyři signály. Každý kanál má svou barvu, se kterou se signál zobrazuje. 
Na každém kanálu může měřit napětí až do 300 V. %(RMS -- efektivní hodnota )

Dále je možnost měřit pomocí digitální sběrnice až 8 digitálních vstupů, přitom digitální vstupy umí pouze zobrazit a měřit časové parametry,  
dekódovat digitální sběrnici umí pouze analogové vstupy. 

Přitom na vodorovné ose se zobrazuje čas, na svislé ose měřené napětí. 

Maximální zobrazované napětí se nastavuje pro každý kanál zvlášť, čas je pro všechny kanály vždy stejný.  

Osciloskop není stavěný pro přesné měření napětí, spíše orientační, protože na vstupu je pouze 8 bitový převodník napětí, ale měří přesně časy.

Všechna ovládací kolečka na osciloskopu se dají také stisknout.

Když podržíte libovolné tlačítko nebo kolečko  2 sekundy, objeví se k němu podrobná nápověda (anglicky). 

\textbf{Menu}  \index{menu osciloskopu} ke každému tlačítku se zobrazuje vždy dole na obrazovce a ovládá se tlačítky pod obrazovkou. 
\label{menu:osc} 

Tlačítko \texttt{Help} (dole uprostřed) zobrazí menu, které  obsahuje mimo jiné položky \texttt{Getting started} a  \texttt{Training singals}. 

\textbf{Panel} \label{panel} osciloskopu je pravá část osciloskopu plná ovládacích prvků.

\textbf{Vzorkovací frekvence} -- počet měření za sekundu. \index{vzorkovací frekvence}
Lze nastavit až 2 GSa/s 
(G -- giga, Sa -- sample = jednotlivé měření ). 

Vzorkovací frekvence se doporučuje nejméně 10x analogová šířka pásma, aby se ze singálu dalo něco poznat. 
Jinými slovy, pro měření signálu o frekvenci 50 kHz potřebuji nastavit vzorkovací frekvenci minimálně 500 kSa/s.

\subsection{Sondy}

\textbf{Sonda} (probe) \index{sonda osciloskopu} je měřící kabel připojený k osciloskopu. 

Sondy připojujte k osciloskopu tak, aby měly stejnou barvu jako kanál, který měří (např. první kanál má žlutou barvu). 

Sondy  mají klobouček s háčkem pro snadné uchopení měřeného drátu. Když se klobouček sundá, lze měřit dotekově hrotem. 
Každá sonda má také pobočný drát zakončený \uv{krokodýlem}. 
Ten se připojuje vždy na zem měřeného obvodu. 
Pokud ho nepřipojíte, bude se vám na kabelu sondy indukovat šum z okolí, který často překryje vlastní signál. 

Červený křížek na sondách slouží ke zkalibrování sond pomocí vestavěného otočného kondenzátoru a signálu \texttt{Probe}, který generuje osciloskop. 
Kalibraci sond obvykle provádět nemusíte, stačí ji udělat při prvním použití sond. 

Sondy jsou obvykle nastavené tak, že dělí vstupní signál 10 (sonda x 10), v osciloskopu se pak nastavuje opětovné vynásobení, 
aby se zobrazovala správná hodnota. 

Osciloskop si pamatuje poslední nastavení, takže při zapnutí není nutné sondy znovu nastavovat.  


\subsection{Zahájení měření}

Skoro všechna tlačítka v této podkapitole najdete na panelu vpravo nahoře v sekci {\tt Run Control}

Tlačítko \texttt{Run/stop} -- červená neměří, zelená měří. Osciloskop ukládá naměřené hodnoty do naplnění paměti, potom nejstarší hodnoty zahazuje a přidává nejnovější. 
Tlačítko \texttt{Single} -- žlutá svítí -- osciloskop udělá právě jednu sadu měření, kterou naplní obsah paměti a dál neměří. 

Měření na sondě se zapíná a vypíná tlačítkem s číslem sondy (mezi velkým a malým otočným kolečkem na panelu dole v sekci {\tt Vertical} ).

Aby se vám měřený signál správně zobrazil, potřebujete mít optimálně nastavené rozlišení jak času, tak napětí.
Pokud si nejste jistí  nastavením osciloskopu, stiskněte tlačítko \texttt{Auto Scale} a osciloskop se pokusí rozlišení nastavit sám. 

Dva signály, které nejsou ze stejných hodin (stejného čipu), se obvykle na monitoru posunují vůči sobě. V takovém případě vypněte 
měření (tlačítko \texttt{Stop}) a změřte Off-line na obrazovce, co potřebujete. 

\subsubsection{Nastavení času}

Všechny ovládací prvky v této podkapitole najdete na panelu osciloskopu vlevo nahoře v sekci {\tt Horizontal}

\begin{itemize}
	\item velké kolečko -- nastavení šířky periody (hodnota pro jeden dílek mřížky se zobrazuje nahoře mírně vpravo )
	\item malé kolečko -- posun zobrazení signálu vlevo nebo vpravo (stisk: návrat do původní polohy) 
	\item tlačítko lupa -- hodně zvětší %todo lze nastavit zvětšení a oblast?
	\item tlačítko \texttt{Horiz} vyvolá \hyperref[menu:osc]{menu} mimo jiné položkami:  %todo odkaz skáče jinam
	\begin{itemize}
		\item \texttt{Time mode} \\ 
		volba \texttt{normal}  -- obvyklé měření
		volba \texttt{XY} umožňuje zobrazit na ose \textit{x} napětí nebo jiné veličiny, \\
		volba \texttt{Roll} nastavuje měření v reálném čase -- vhodné pro pomalá měření napětí (točení potenciometrem ) 

		\item \texttt{Time Ref} -- počátek měření času je umístěn vlevo, na střed, vpravo
	\end{itemize} 	
\end{itemize}

\subsubsection{Nastavení napětí}

Ovládací kolečka z této podkapitoly najdete na panelu osciloskopu  dole v sekci {\tt Vertical} ).
Logika jejich ovládání je podobná jako u měření času. 
\begin{itemize}
	\item velké kolečko -- zesílení signálu (hodnota pro jeden dílek mřížky se zobrazuje nahoře vlevo )
	\item malé kolečko -- posun zobrazení signálu nahoru nebo dolů, tzv. offset \index{offset} (stisk: návrat do původní polohy) 
	 
\end{itemize}

\subsection{Trigger} %todo pro signál cviční sinusovka to na vzestupné hraně funguje a na sestupné to dělá paseku 

Nastavení, od kdy přesně má osciloskop začít měřit, je v mnoha případech klíčové. 

\textbf{Trigger} -- říká: teď začni měřit. Trigger může mít pouze jeden vstup, který lze velmi různě navolit pomocí tlačítka \texttt{Trigger}.

tlačítko Mode Coupling %todo doplnit
tlačítko Force Trigger -- okamžitě zahájí měření (v normal módu) 


\subsection{Význam některých tlačítek -- heslovitě}



Lze uložit až 10 svých nastavení osciloskopu a podle potřeby se k nim vracet.  %todo  jak ? ******************

 Tlačítko  \texttt{Wave  gen} -- modře svítí -- zapnuto / nesvítí vypnuto. 
 Generátor funkcí je popsán v nápovědě osiloskopu (stiskněte \texttt{Wave  gen} a držte 2 sekundy). 

 Tlačítko \texttt{Meas}  -- menu měření.
Zde si nastavíte, co všechno chcete měřit  (až 4 veličiny zaráz).

Tlačítko \texttt{Cursors}   (=pravítka, dvě vodorovná a dvě svislá)  -- lze tím měřit zcela manuálně cokoliv na obrazovce.
Nejčastější použití -- sledování PWM a signálů na sběrnicích. 

Tlačítko \texttt{Refs}  (referenční signály) -- umí si pamatovat dva signály a srovnávat s nimi aktuální průběh  %todo ************************ jak se s tím pracuje 

Tlačítko \texttt{Math}  umí arit. výpočty se dvěma signály, taky umí Fourrierovu analýzu signálu. 

Tlačítko  \texttt{Digital} -- nastavení měření na digitálních vstupech.

Tlačítko   \texttt{Serial} -- serial decode mode  
%Tlačítko/menu ???? \texttt{} treshold --při dekódování sběrnic $\rightarrow$ nastavím, kde končí log. O a Log. 1  



\subsection{Ostatní}

\subsubsection{Menu sondy} %todo jak se leze do menu sondy ??

\textbf{probe}: dělička v sondě (musí se nastavit stejně jak na sondě), obvyklá hodnota je 10:1

zapnu střídavou vazbu: odstraní stejnosměrnou složku signálu %todo ????

\textbf{invert}: zobrazuje kladnou složku dolů místo nahoru 

\textbf{BW limit} potlačuje signály nad 20 MHz (tuto hodnotu na tomto osc. nelze měnít) $\rightarrow$ redukuje šumy, které nás obvykle nezajímají 


%--------------

%Některé položky z menu:

%\begin{description}
%	\texttt{Mode}: \texttt{auto} -- čeká \uv{nějakou dobu} na signál z triggeru a pak začne měřit  
%	\texttt{normal} -- čeká na signál z triggeru neomezeně dlouho 
%	
%	
%	\item  \texttt{Coupling DC} -- měří všechno 
%	
%	\item  \texttt{Coupling AC} -- potlačí stejnosměrné složky signálu 
%	
%	\item \texttt{LF Reject} -- potlačení pomalých frekvencí 
%	
%	
%\end{description} 


%\texttt{HF Reject} -- potlačení vysokofrekvenční složky 

%\texttt{Noise Rej} -- potlačení šumu   %nepoznal jsem rozdíl  - pouze na "rozumně tvrdém" signálu 

%\texttt{Holdoff} -- nastavení doby, po kterou osciloskop ignoruje další signály %todo z triggeru nebo celkově? 

%\texttt{treshold} -- nastavení úrovně, jejíž překročení spouští měření






