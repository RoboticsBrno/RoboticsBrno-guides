\section*{Příloha A~-- Hodnoty vybraných součástek}


\hypertarget{1N4148}{}
{\bf Dioda 1N4148}  \index{dioda!1N4148}\index{1N4148!dioda}

Maximální napětí: 100 V

Maximální proud: 200 mA

Maximální výkon:  500 mW

Úbytek napětí: 1,8 V

\vskip 2mm

\hypertarget{1N4007}{}
{\bf Dioda 1N4007} \index{dioda!1N4007}\index{1N4007!dioda}

Maximální napětí: 1000 V

Maximální proud: 1 A

Maximální výkon: 3 W

Úbytek napětí: 1,1 V

\vskip 2mm

\hypertarget{BCC337}{}
{\bf tranzistor BC337} \index{tranzistor!BCC337}\index{BCC337!tranzistor}

Max. napětí mezi kol. a~emit. $V \rm _{CEO}$:50 V

Max. napětí mezi bází a~emit. $V \rm _{CBO}$: 5 V

Max. proud tekoucí kolektorem $I \rm _C$: 800 mA

Maximální výkon $ P \rm _C$: 625 mW

Zesílení $ h \rm _{fe}$: 100 až 600

\vskip 2mm

\hypertarget{BCC547}{}
{\bf tranzistor BC547} \index{tranzistor!BCC547}\index{BCC547!tranzistor}

Max. napětí mezi kol. a~emit. $ V \rm_{CEO}$: 45 V

Max. napětí mezi bází a~emit. $ V \rm _{CBO}$: 6 V
 
Max. proud tekoucí kolektorem $ I \rm _C$: 100 mA

Maximální výkon $P \rm _C$: 500 mW

Zesílení $h \rm _{fe}$: 110 až 800

\vskip 2mm

\hypertarget{BD911}{}
{\bf tranzistor BD911}  \index{tranzistor!BD911}\index{BD911!tranzistor}

Max. napětí mezi kol. a~emit.$V \rm _{CEO}$: 100 V

Max. napětí mezi bází a~emit. $V \rm _{CBO}$: 5 V

Max. proud tekoucí kolektorem $I \rm _C$: 15 A

Maximální výkon $P \rm _C$: 90 W

Zesílení $h \rm _{fe}$: 5 až 250

Důležité je si všimnout, že minimální zesílení\footnote{Platí pokud bude kolektorem protékat 10~A.}  je 5. 
To znamená, že pokud budeme chtít tranzistorem spínat proud 10~A, tak budeme muset nechat bází téct řídící proud 2~A,
 to znamená, že ho nemůžeme naplno využít, pokud ho budeme řídit mikrokontrolérem, tj. musíme ho spínat jiným tranzistorem. 
 Anebo přijdeme na myšlenku, že pokud budeme chtít řídit velké proudy, budeme potřebovat tranzistory typu MOS-FET, např.: IRF520, IRL3803.

%Maximální proud tekoucí kolektorem $I_C$\footnote{Ang.: Collector Current}

%Maximální výkon $P_C$\footnote{Ang.: Collector Power Dissipation}

%Zesílení $h_{fe}$\footnote{Ang.: DC Current Gain}


\section*{Příloha B -- Vývojový deník robota pro Ketchup House - stručná historie výroby }

\begin{description}
	\item[září - říjen:] 
	
	\begin{itemize}
	    \item[]
		\item první návrh strategie
		\item návrh mechaniky a konstrukce
		\item  vize ohledně senzorů
		\item jako řídící jednotka byla zvolena deska Arduino Mega, především kvůli velkému počtu pinů, cíl: vystačit si s jednou deskou  
		
	\end{itemize} 
	
	\item[listopad:] 
	
	\begin{itemize}
		\item[]
		\item druhý návrh strategie
		\item výroba první konstrukce, jsou použity vrtačkové motory
		\item pokusy o zprovoznění/naprogramování driverů pro motory, byly použity VNH32, které známe ze starších strojů, 
		tyto drivery se vypínaly kvůli nízkému napětí baterií + byla potřeba úprava driverů (odpájení Zenerovy diody)  
		\item motory se ukazují jako příliš silné a obtížně řiditelné 
	\end{itemize}
	
	\item[prosinec:]
	\begin{itemize}
		\item[]
		\item druhá konstrukce, místo vrtačkových byly použity čínské modelářské motory
	\end{itemize} 
	
	\item[leden:]
	\begin{itemize}
		\item[]
		\item  pololetní klasifikace
	\end{itemize}  
	
	\item[únor:] 
	\begin{itemize}
		\item[]
		\item osazování senzorů QRD1114 pro čáry
		\item první verze programuQRD1114 z Číny se neosvědčily $\rightarrow$ znova koupit a zapájet senzory z GM Electonic
		
		\item řešení problémů s chybným zapojením senzorů 
		\item kalibrace senzorů 
		\item QRD1114 z Číny se neosvědčily $\rightarrow$ znova koupit a zapájet senzory z GM Electonic		
	\end{itemize} 
	
	\item[březen:] 
	\begin{itemize}
		\item[]
		\item zprovoznění serva pro vypouštění plechovek
		\item testování senzorů HS-04 pro ultrazvukovou detekci soupeře 
	\end{itemize}  
	
	\item[duben:] 
	\begin{itemize}
		\item[]
		\item čtení z ultrazvuků je příliš pomalé a robot nestíhá detekovat čáru $\rightarrow$
		přidání druhé ATmega desky pro čtení ultrazvuků a enkodérů
		\item rozchození komunikace mezi deskami 
		\item rozchození enkodérů $\rightarrow$ robot konečně jede rovně 
	\end{itemize}  
	
	\item[květen:] 
	\begin{itemize}
		\item[]
		\item QRD nečtou správně $\rightarrow$ posun QRD níž
		\item přechod na algoritmus typu stavový automat
		\item robot pořád nechce jet po čáře
		\item robot  se hodně zpomaluje, pokud veze více plechovek najednou (se čtyřmi plechovkami už stojí) $\rightarrow$ snaha o programové řešení pomocí zvýšení výkonu motorů při zpomalení robota
	\end{itemize}
	
	
	\item[červen:] 
	\begin{itemize}
		\item[]
		\item intenzivní snaha dodělat a naprogramovat robota, provázená průběžně poruchami všeho druhu;
		tyto poruchy byly často způsobeny nezkušeností a nedotaženými detaily z předchozích měsíců  
		\item do termínu soutěže 10.6. se nepodařilo naprogramovat potřebné schopnosti robota, robot se přesto soutěže zúčastnil
		\item v den soutěže se spálila základní deska a její výměna vzala skoro veškerý zbývající čas plánovaný na dotažení programu  
	\end{itemize} 
\end{description}
	
{\bf Zkušenosti a závěry}
	\begin{itemize}
		\item[]
		\item  skutečná stavba a programování trvá nejméně 2x tak dlouho, než předpokládá plán 
		\item (ze zkušenosti i s jinými roboty): vždy se naplánuje několik verzí obtížnosti a prakticky vždy se postaví a zprovozní pouze ta nejjednodušší, více se nestihne, především naprogramovat  
		\item je potřeba buď mít dva stejné roboty nebo vézt celého robota s sebou na soutěž ještě jednou v náhradních dílech; kdybychom neměli náhradní základní desku, tak jsme skončili, ještě než soutěž začala  
	\end{itemize}



\section*{Příloha C -- Poznámky a~vize}



\subsection*{Přerušení}

Co je přerušení? Procesor může zvládnout pouze jednu operaci na jeden tik krystalu, 
postupuje od jednoho příkazu k~druhému a~nemůže jen tak všeho nechat a~věnovat se něčemu jinému, občas je to ale potřeba.
 Při přerušení procesor všeho nechá a~bude se věnovat přerušení, potom co skončí se bude věnovat dál programu tam, kde přestal. 

\subsection*{Baterie}

Dobíjecí: napětí 1,2 V~\index{baterie!dobíjecí}\index{dobíjecí!baterie}

Jednorázové: napětí 1,5 V~\index{baterie!jednorázové}\index{jednorázové!baterie}
 

\subsection*{Vize -- co přidat do textu}

Adruino IDE 

Laser ve Lablabu 

Baterie LiON a~jejich nabíjení 

Převodník napěťových úrovní 

Pájení 

Sběrnice - USART/UART, IIC, SPI, další?

%todo 
% - modrozub

% - propojení s mobilem 

% - verze jako poslední příloha 
% součástky: tlačítko včetně zapojení, barevný senzor 
%seznam literatury 

% % pokud nemáte hodně vývodů na desce, potřebujete nepájivé pole po rozvedení napájení (GND, 5V nebo 3V3). 
 % verzování do VSCode 
% jak přesně funguje funkce map 
% otestovat řízení serva a zkontrolovat piny GND a data 


