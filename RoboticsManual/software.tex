 \chap Software
 
 
\sec \TeX

\secc Proč používat \TeX

Tento text je psán v~sázecím systému \ii TeX \TeX . Jeho silná stránka je především matematická sazba 
(bohužel nevyužijeme) a~snadné zpracování obsahu, rejstříků, seznamů obrázků a~tabulek a~podobně, což dramaticky urychluje přípravu dokumentu. Čas, který vložíte do nastavení a~učení se systému, se vrátí v~rychlosti práce $\rightarrow$ jedná se o řešení vhodné pro delší dokumenty, např. pro soutěž SOČ nebo dlouhodobou maturitní práci. Existují různé nadstavby a~rozšíření \TeX{}u, nejznámější je \LaTeX{}.
 

\secc Literatura 

\begitems \style O
* \TeX pro pragmatiky: \url{http://petr.olsak.net/ftp/olsak/tpp/tpp.pdf}
* První setkání se \TeX{}em: \url{http://petr.olsak.net/ftp/cstex/doc/prvni.pdf} 
* Tipy pro OPmac: \url{http://petr.olsak.net/opmac-tricks.html}
* \LaTeX pro pragmatiky: \url{http://mirrors.nic.cz/tex-archive/info/czech/\|latex-pro-pragmatiky/latex-pro-pragmatiky.pdf} \emergencystretch=2cm
\enditems

Příkazy \TeX{}u i~\LaTeX{}u podobně jako C++ a~systémy typu Linux {\bf rozlišují velká a~malá písmena}. 
 

\secc Editory 

Ve WinXP používám PSpad, v~linuxu TeX Maker. Oba editory umí zavolat překlad do pdf pomocí klávesové zkratky, zobrazit výsledný pdf a~barevné zvýraznění syntaxe. PSpad se dá rozjet i~pod linuxem pomocí prostředí Wine, ale nepodařilo se mi rozchodit volání překladu. 

\secc Čeština 

Ve WinXP používám kódování CP1250. V~linuxu je nativní UTF-8. Při přechodu z~Win na Linux není problém. Při přechodu z~linuxu na WinXP se vše zobrazuje správně, ale při překladu některé znaky (ž) chybně. Pomohlo ruční nastavení kódování (menu {\it Formát}) z~CP1250 a~zpět kombinované s~opakovanými překlady, vše v~PSpadu.   
 
\sec Další software 

\secc Proficad 

Proficad\fnote{Instalační soubor seženete v~kroužku nebo na webu.} je software určený původně pro snadné a~rychlé kreslení elektronických schémat a~v~této oblasti je vynikající. Lze jej použít i~jako jednoduchý vektorový editor obrázků. \ii Proficad 

SPŠ Sokolská zakoupila plnou multilicenci pro Proficad, takže studenti i~učitelé jej mohou používat bez omezení. 

Ovládání programu je velice intuitivní a~nápovědu prakticky nepotřebujete -- s~jedinou výjimkou, a~tou je nastavení rastru. Po instalaci je rastr zobrazení automaticky nastaven na 2 mm. To znamená, že součástky můžete umisťovat například 10 mm nebo 12 mm od kraje, ale nic mezi tím. Většinou se to hodí -- součástky máte na schématu pěkně zarovnané -- ale někdy je prostě potřeba rastr například vypnout neboli nastavit na nulu. Nastavení rastru je schované zde:  {\it soubor/nastavení/dokument/obsah/rastr}.

\secc Prohlížení Pdf

Ve WinXP se osvědčil Foxit Reader verze 2.3, v~linuxu Evince.  

