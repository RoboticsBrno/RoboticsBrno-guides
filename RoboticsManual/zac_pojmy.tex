\sec Základní pojmy -- čipy

\secc{Datasheet}

{\bf Datasheet} \ii{datasheet} je dokument, ve kterém jsou detailně popsány vlastnosti a možnosti dané elektronické součástky. Každá součástka má svůj datasheet. 

Datasheet pro každou součástku je možné najít na internetu\fnote{například na stránce \url{www.datasheetcatalog.com}}
 nebo na stránkách výrobce nebo prodejce součástky. Bohužel jsou všechny datasheety anglicky.  

\secc{Mikroprocesor, mikrokontrolér} 

{\bf Mikroprocesor, mikrokontrolér, čip} \ii{mikroprocesor} \ii{mikrokontrolér} 
 znamenají totéž -- integrovaný obvod, který se snažíme naprogramovat, aby řídil robota nebo jeho část.\label[cip1] 

{\bf Pin} \ii{pin} je vývod (nožička) čipu. Jednoduché čipy (např. ATtiny) mají osm pinů, složitější čipy mají 32, 40, nebo také 100 pinů.

Pin může být nastavený jako vstupní nebo jako výstupní. 

Pokud je pin nastavený jako {\bf vstupní}, umí určit, zda na něm je napětí odpovídající logické jedničce\ii logická~jednička (5~V nebo 3,3~V podle typu čipu) nebo logické nule (0~V)\ii logická~nula . Také umí přečíst, jaké je na něm analogové napětí. \rfc{analogové} 

Pokud je pin nastavený jako {\bf výstupní},  umí se nastavit na logickou jedničku nebo logickou nulu. \rfc{odpovídající funce čipu} 

\secc{Bity a bajty}

Každá číslice ve dvojkové soustavě reprezentuje jeden {\bf bit} \ii{bit} (nabývá hodnot 0 nebo 1). 

Osm bitů dohromady tvoří \ii{bajt} {\bf bajt}. Nejnižší bit v bajtu leží vpravo (tzv. {\bf nultý bit}\ii bit/nultý ),  další je nalevo od něj (první bit), až do sedmého bitu, který je nejvíce vlevo.




